\documentclass[11pt,a4paper,twocolumn]{article}

% Packages
\usepackage[margin=1in]{geometry}
\usepackage{cite}
\usepackage{amsmath,amssymb,amsfonts}
\usepackage{graphicx}
\usepackage{xcolor}
\usepackage{hyperref}
\usepackage{listings}
\usepackage{url}
\usepackage{titlesec}
\usepackage{abstract}

% Section formatting
\titleformat{\section}{\Large\bfseries}{\thesection}{1em}{}
\titleformat{\subsection}{\large\bfseries}{\thesubsection}{1em}{}
\titleformat{\subsubsection}{\normalsize\bfseries}{\thesubsubsection}{1em}{}

% Code listing settings
\lstset{
    basicstyle=\ttfamily\small,
    breaklines=true,
    frame=single,
    language=Python,
    showstringspaces=false,
    commentstyle=\color{gray},
    keywordstyle=\color{blue},
    stringstyle=\color{red},
    numbers=left,
    numberstyle=\tiny\color{gray}
}

% Hyperref settings
\hypersetup{
    colorlinks=true,
    linkcolor=blue,
    filecolor=magenta,
    urlcolor=cyan,
    citecolor=blue
}

% Abstract styling
\renewcommand{\abstractnamefont}{\Large\bfseries}
\renewcommand{\abstracttextfont}{\normalsize}

\begin{document}

% Title
\title{\textbf{COVID-19 Global Analytics Platform\\Using AWS Cloud Services}}

\author{Cloud Computing Project}

\date{}

\twocolumn[
\begin{@twocolumnfalse}
\maketitle

\begin{abstract}
This project implements a comprehensive cloud-based analytics platform for analyzing global COVID-19 pandemic data across 235 countries and territories. The solution leverages AWS serverless services including S3, Athena, and Glue to process 429,435 data records spanning from January 2020 to August 2024. Using Infrastructure-as-Code principles with AWS CDK, the platform enables data-driven analysis of cases, deaths, vaccinations, and demographic correlations, providing actionable insights for public health decision-making.

\textbf{Data Source} \cite{owid2024}: \url{https://raw.githubusercontent.com/owid/covid-19-data/master/public/data/owid-covid-data.csv}

\textbf{Platform}: AWS S3, AWS Athena, AWS Glue, Metabase

\textbf{Keywords}: AWS, Cloud Computing, COVID-19 Analytics, Serverless, Data Analytics, Infrastructure-as-Code

\textbf{Github}:  \url{https://github.com/aaditya-diwan/covid-19-Analytics}
\end{abstract}

\vspace{0.5cm}
\end{@twocolumnfalse}
]

% Main content starts here
\section{Application Overview}
This project implements a cloud-native analytics platform for analyzing global COVID-19 pandemic data using AWS serverless services. The system processes comprehensive datasets from Our World in Data (OWID), encompassing 429,435 records across 255 unique locations, tracking cases, deaths, vaccinations, and demographic factors from January 2020 through August 2024.

The platform architecture leverages AWS S3 for scalable cloud storage, AWS Glue for automated schema discovery and metadata management, and AWS Athena for serverless SQL query execution. Infrastructure deployment utilizes AWS Cloud Development Kit (CDK) with TypeScript, implementing Infrastructure-as-Code principles for reproducible and maintainable cloud resource provisioning. The solution integrates with Metabase for interactive data visualization, enabling stakeholders to explore pandemic trends through customizable dashboards.

This serverless approach eliminates infrastructure management overhead while providing elastic scalability, pay-per-query economics, and high availability for data-driven public health analysis.

\section{Objectives}

\subsection{Primary Objectives}
\begin{itemize}
    \item \textbf{Cloud-Native Data Storage}: Implemented - Successfully deployed AWS S3 buckets with versioning, encryption, and lifecycle management for 100+ MB COVID-19 datasets

    \item \textbf{Serverless Query Engine}: Implemented - Configured AWS Athena workgroup enabling SQL analytics without managing database infrastructure

    \item \textbf{Automated Schema Discovery}: Implemented - Deployed AWS Glue crawler for automatic table schema generation from CSV data sources

    \item \textbf{Infrastructure-as-Code}: Implemented - Developed AWS CDK stacks in TypeScript for reproducible infrastructure deployment

    \item \textbf{Multi-Dimensional Analytics}: Implemented - Created 30+ SQL queries analyzing cases, deaths, vaccinations across geographic and demographic dimensions
\end{itemize}

\subsection{Secondary Objectives}
\begin{itemize}
    \item \textbf{Data Processing Pipeline}: Implemented - Python scripts for automated data download, cleaning, and transformation using Pandas

    \item \textbf{Cost Optimization}: Implemented - S3 lifecycle policies for query result retention and Athena query size limits (1GB threshold)

    \item \textbf{Interactive Visualization}: Designed - Metabase dashboard configuration for exploring pandemic trends with filters and drill-down capabilities
\end{itemize}

\section{Problem Description and Dataset}

\subsection{Problem Significance}
The COVID-19 pandemic represents one of the most significant public health crises in modern history, requiring comprehensive data analytics infrastructure for informed decision-making:

\begin{itemize}
    \item \textbf{Global Scale}: 693.7 million confirmed cases and 6.2 million deaths across 235 countries as of August 2024

    \item \textbf{Data Complexity}: Multi-dimensional data spanning epidemiological metrics, vaccination campaigns, and demographic factors

    \item \textbf{Timeliness Requirements}: Need for rapid analysis to support policy decisions and public health interventions

    \item \textbf{Accessibility}: Requirement for interactive tools enabling non-technical stakeholders to explore pandemic data
\end{itemize}

\subsection{Dataset Characteristics}
The analysis utilizes the Our World in Data (OWID) COVID-19 dataset with comprehensive global coverage:

\begin{itemize}
    \item \textbf{Source}: Our World in Data COVID-19 Dataset \cite{owid2024}
    \item \textbf{Volume}: 429,435 daily records (255 locations × 1,688 days)
    \item \textbf{File Size}: 100.5 MB CSV format
    \item \textbf{Time Period}: January 1, 2020 to August 31, 2024
    \item \textbf{Geographic Coverage}: 235 individual countries plus 20 regional aggregates
    \item \textbf{Data Completeness}: 77.9\% overall
    \item \textbf{Key Metrics}:
    \begin{itemize}
        \item Epidemiological: total\_cases, new\_cases, total\_deaths, new\_deaths
        \item Vaccination: total\_vaccinations, people\_vaccinated, people\_fully\_vaccinated
        \item Demographics: population, population\_density, median\_age, aged\_65\_older
        \item Identifiers: iso\_code, continent, location, date
    \end{itemize}
\end{itemize}

\section{Methodology and Implementation}

\subsection{Technical Architecture}
The system employs a serverless cloud architecture optimized for analytics workloads:

\begin{itemize}
    \item \textbf{Storage Layer}: AWS S3 with versioning and server-side encryption (SSE-S3)
    \item \textbf{Catalog Layer}: AWS Glue Data Catalog with automated crawler for schema discovery
    \item \textbf{Query Layer}: AWS Athena workgroup with Presto SQL engine
    \item \textbf{Orchestration Layer}: Python scripts for ETL pipeline execution
    \item \textbf{Visualization Layer}: Metabase connected via Athena JDBC driver
    \item \textbf{Infrastructure Layer}: AWS CDK v2.215.0 with TypeScript for IaC deployment
\end{itemize}

\subsection{AWS Cloud Services}

\subsubsection{AWS S3 Storage Infrastructure}
Two S3 buckets provide segregated storage for data and query results:

\begin{itemize}
    \item \textbf{Data Bucket} (covid-analytics-data-\{account-id\}):
    \begin{itemize}
        \item Versioning enabled for data recovery
        \item Server-side encryption with S3-managed keys
        \item Lifecycle rule: Delete old versions after 30 days
        \item Directory structure: raw-data/, processed/
    \end{itemize}

    \item \textbf{Athena Results Bucket} (covid-analytics-athena-\{account-id\}):
    \begin{itemize}
        \item Stores query execution outputs
        \item Lifecycle rule: Delete results after 7 days
        \item Reduces storage costs for transient query data
    \end{itemize}
\end{itemize}

\subsubsection{AWS Glue Data Catalog}
Automated metadata management for schema-on-read analytics:

\begin{itemize}
    \item \textbf{Database}: covid\_data\_warehouse
    \item \textbf{Table}: owid\_covid\_data (18 columns, CSV SerDe)
    \item \textbf{Crawler}: covid-data-crawler
    \begin{itemize}
        \item IAM Role: GlueCrawlerRole with S3 read permissions
        \item Schedule: On-demand execution
        \item Target: s3://covid-analytics-data-*/raw-data/
        \item Schema change policy: UPDATE\_IN\_DATABASE
    \end{itemize}
    \item \textbf{Schema Discovery}: Automatic column type inference from CSV headers
\end{itemize}

\subsubsection{AWS Athena Query Service}
Serverless SQL analytics with performance controls:

\begin{itemize}
    \item \textbf{WorkGroup}: covid-analytics-workgroup
    \item \textbf{Query Engine}: Presto-based (Engine Version: AUTO)
    \item \textbf{Result Location}: Athena results S3 bucket
    \item \textbf{Encryption}: SSE-S3 for query results
    \item \textbf{Cost Controls}: 1GB bytes scanned limit per query
    \item \textbf{Monitoring}: CloudWatch metrics enabled for query performance tracking
    \item \textbf{Pricing Model}: \$5 per TB scanned (\$0.01-\$0.05 per typical query)
\end{itemize}

\subsection{Infrastructure-as-Code Implementation}

\subsubsection{AWS CDK Stack Architecture}
The infrastructure deployment consists of three modular CDK stacks:

\begin{lstlisting}[language=bash]
infrastructure/
|-- bin/infrastructure.ts    # CDK app
|-- lib/
|   |-- storage-stack.ts     # S3 buckets
|   |-- glue-stack.ts        # Data catalog
|   |-- athena-stack.ts      # Query workgroup
|-- package.json             # CDK v2.215.0
\end{lstlisting}

\textbf{StorageStack} creates S3 infrastructure:
\begin{lstlisting}
const dataBucket = new s3.Bucket(this,
  'CovidDataBucket', {
    versioned: true,
    encryption:
      s3.BucketEncryption.S3_MANAGED,
    lifecycleRules: [{
      noncurrentVersionExpiration:
        cdk.Duration.days(30)
    }]
});
\end{lstlisting}

\textbf{GlueStack} configures data catalog:
\begin{lstlisting}
const database = new glue.CfnDatabase(
  this, 'CovidDatabase', {
    catalogId: cdk.Aws.ACCOUNT_ID,
    databaseInput: {
      name: 'covid_data_warehouse'
    }
});

const crawler = new glue.CfnCrawler(
  this, 'CovidCrawler', {
    role: crawlerRole.roleArn,
    databaseName: database.ref,
    targets: {
      s3Targets: [{
        path: `s3://${dataBucket}/raw-data/`
      }]
    }
});
\end{lstlisting}

\textbf{AthenaStack} provisions query infrastructure:
\begin{lstlisting}
const workgroup = new athena.CfnWorkGroup(
  this, 'CovidWorkGroup', {
    name: 'covid-analytics-workgroup',
    workGroupConfiguration: {
      resultConfiguration: {
        outputLocation:
          `s3://${athenaResultsBucket}/`,
        encryptionConfiguration: {
          encryptionOption: 'SSE_S3'
        }
      },
      bytesScannedCutoffPerQuery:
        1024 * 1024 * 1024, // 1GB limit
      publishCloudWatchMetricsEnabled: true
    }
});
\end{lstlisting}

\subsubsection{CDK Deployment Workflow}
\begin{lstlisting}[language=bash]
# Install dependencies
npm install

# Synthesize CloudFormation template
npx cdk synth

# Deploy infrastructure
npx cdk deploy --all

# View stack differences
npx cdk diff
\end{lstlisting}

\subsection{Data Processing Pipeline}

\subsubsection{Data Acquisition Script}
Python automation for dataset download:

\begin{lstlisting}
import requests
from datetime import datetime

def download_owid_data():
    url = "https://raw.githubusercontent.com/" \
          "owid/covid-19-data/master/" \
          "public/data/owid-covid-data.csv"

    response = requests.get(url)
    timestamp = datetime.now()
      .strftime("%Y%m%d_%H%M%S")

    output_path = f"data/raw/owid/" \
                  f"owid-covid-data-" \
                  f"{timestamp}.csv"

    with open(output_path, 'wb') as f:
        f.write(response.content)

    print(f"Downloaded: {output_path}")
    print(f"Size: {len(response.content)
                    / (1024*1024):.2f} MB")
\end{lstlisting}

\subsubsection{Data Processing Script}
Pandas-based ETL for data cleaning:

\begin{lstlisting}
import pandas as pd

def process_owid_data(input_file):
    # Load and parse CSV
    df = pd.read_csv(input_file)
    df['date'] = pd.to_datetime(df['date'])

    # Select essential columns (16 of 61)
    columns = [
        'iso_code', 'continent', 'location',
        'date', 'total_cases', 'new_cases',
        'total_deaths', 'new_deaths',
        'total_vaccinations',
        'people_vaccinated',
        'people_fully_vaccinated',
        'new_vaccinations', 'population',
        'population_density', 'median_age',
        'aged_65_older'
    ]
    df = df[columns]

    # Remove rows with missing key fields
    df = df.dropna(subset=[
        'location', 'date'])

    # Save processed data
    output = "data/processed/owid/" \
             "owid-covid-processed.csv"
    df.to_csv(output, index=False)

    print(f"Processed {len(df)} records")
    print(f"Output: {output}")
    return df
\end{lstlisting}

\subsubsection{Aggregation Pipeline}
Creating analytical summary tables:

\begin{lstlisting}
def create_aggregated_tables(input_file):
    df = pd.read_csv(input_file)

    # Daily summary by location
    daily = df.groupby([
        'location', 'date']).agg({
        'total_cases': 'max',
        'new_cases': 'sum',
        'total_deaths': 'max',
        'new_deaths': 'sum',
        'total_vaccinations': 'max',
        'people_fully_vaccinated': 'max',
        'population': 'first'
    }).reset_index()

    # Calculate derived metrics
    daily['vaccination_rate'] = (
        daily['people_fully_vaccinated'] /
        daily['population'] * 100
    )
    daily['mortality_rate'] = (
        daily['total_deaths'] /
        daily['total_cases'] * 100
    )

    # Save aggregated data
    output = "data/processed/aggregated/" \
             "daily_summary.csv"
    daily.to_csv(output, index=False)

    return daily
\end{lstlisting}

\subsection{SQL Query Analytics}

\subsubsection{Global Pandemic Statistics}
Query for worldwide cases:

\begin{lstlisting}[language=SQL]
WITH latest_cases AS (
    SELECT
        location,
        MAX(total_cases) as cases
    FROM covid_data
    WHERE location NOT IN (
        'World',
        'Africa', 'Asia', 'Europe', 'North America', 'South America', 'Oceania',
        'High income', 'Low income', 'Lower middle income', 'Upper middle income',
        'European Union (27)',
        'High-income countries',
        'Upper-middle-income countries',
        'Lower-middle-income countries',
        'Low-income countries'
    )
    AND total_cases IS NOT NULL
    GROUP BY location
)
SELECT
    SUM(cases) as total_cases_all_time
FROM latest_cases;
-- Result: 693,740,129 cases
--         6,225,038 deaths
--         235 countries
\end{lstlisting}

\subsubsection{Country Rankings with Normalization}
Top countries by cases per 100,000 population:

\begin{lstlisting}[language=SQL]
SELECT
    location,
    MAX(total_cases) AS total_cases,
    MAX(population) AS population,
    (MAX(total_cases) /
     NULLIF(MAX(population), 0)
     * 100000) AS cases_per_100k
FROM owid_covid_data
WHERE population > 0
    AND location NOT LIKE '%income%'
    AND location NOT IN (
        'World', 'England', 'Scotland')
GROUP BY location
ORDER BY cases_per_100k DESC
LIMIT 30
\end{lstlisting}

\subsubsection{Vaccination Progress Analysis}
Tracking vaccination effectiveness by brackets:

\begin{lstlisting}[language=SQL]
SELECT
    CASE
        WHEN vaccination_rate < 20
            THEN '0-20%'
        WHEN vaccination_rate < 40
            THEN '20-40%'
        WHEN vaccination_rate < 60
            THEN '40-60%'
        WHEN vaccination_rate < 80
            THEN '60-80%'
        ELSE '80-100%'
    END AS vaccination_bracket,
    AVG(daily_cases_per_100k)
      AS avg_daily_cases,
    AVG(mortality_rate)
      AS avg_mortality_rate
FROM (
    SELECT
        location,
        date,
        (people_fully_vaccinated /
         population * 100)
           AS vaccination_rate,
        (new_cases / population * 100000)
           AS daily_cases_per_100k,
        (total_deaths /
         NULLIF(total_cases, 0) * 100)
           AS mortality_rate
    FROM owid_covid_data
    WHERE people_fully_vaccinated IS NOT NULL
        AND total_cases > 1000
)
GROUP BY vaccination_bracket
ORDER BY vaccination_bracket
\end{lstlisting}

\subsubsection{Time Series Trend Analysis}
Cumulative cases over time for major countries:

\begin{lstlisting}[language=SQL]
SELECT
    date,
    location,
    total_cases
FROM owid_covid_data
WHERE location IN (
    'United States', 'India', 'Brazil',
    'United Kingdom', 'Germany', 'France',
    'Italy', 'Spain', 'Canada', 'Japan'
)
ORDER BY location, date
-- Output: Time series for line charts
-- Visualization: Multi-line chart
\end{lstlisting}

\section{Suitability of AWS Serverless Architecture}
AWS serverless services demonstrated exceptional suitability for pandemic data analytics:

\begin{itemize}
    \item \textbf{Elastic Scalability}: Athena automatically scales to query datasets from MB to PB without capacity planning

    \item \textbf{Cost Efficiency}: Pay-per-query pricing (\$0.01-\$0.05 per query) eliminates idle database costs

    \item \textbf{Zero Administration}: No database servers to provision, patch, or maintain

    \item \textbf{High Availability}: AWS-managed services provide 99.9\% SLA without manual failover configuration

    \item \textbf{Schema Flexibility}: Glue crawler adapts to schema changes in source data automatically

    \item \textbf{Infrastructure-as-Code}: CDK enables version-controlled, reproducible deployments across environments
\end{itemize}

\section{Software Features}

\subsection{Analytical Capabilities}

\subsubsection{Epidemiological Analytics}
\begin{itemize}
    \item Global and country-level case/death tracking
    \item Daily new cases and deaths time series
    \item 7-day rolling averages for trend smoothing
    \item Case fatality rate (CFR) calculations
    \item Attack rate analysis (percentage of population infected)
\end{itemize}

\subsubsection{Vaccination Campaign Analysis}
\begin{itemize}
    \item Vaccination progress tracking (doses administered, people vaccinated, fully vaccinated)
    \item Vaccination rate calculations by country
    \item Correlation analysis between vaccination levels and case rates
    \item Timeline comparison of vaccination rollout speeds across nations
\end{itemize}

\subsubsection{Demographic Correlation}
\begin{itemize}
    \item Age-based severity analysis (median age vs. mortality)
    \item Population density impact on transmission rates
    \item Elderly population (65+) vulnerability assessment
    \item Socioeconomic factors (GDP per capita correlation)
\end{itemize}

\subsubsection{Geographic Analysis}
\begin{itemize}
    \item Continental breakdown of pandemic impact
    \item Country rankings by multiple metrics
    \item Per-capita normalization for fair cross-country comparison
    \item Regional aggregate statistics
\end{itemize}

\subsection{Technical Features}

\subsubsection{Data Quality Management}
\begin{itemize}
    \item Automated data validation (age ranges, date consistency)
    \item NULL value handling with NULLIF() functions
    \item Duplicate detection and regional aggregate filtering
    \item Data completeness reporting (77.9\% overall)
\end{itemize}

\subsubsection{Query Optimization}
\begin{itemize}
    \item Efficient aggregation patterns (MAX for cumulative metrics)
    \item WHERE clause filtering to reduce data scanned
    \item Query result caching for repeated analyses
    \item 1GB scan limit to prevent runaway query costs
\end{itemize}

\subsubsection{Visualization Integration}
\begin{itemize}
    \item Metabase JDBC connection to Athena
    \item Interactive dashboard with filters (date range, location, continent)
    \item Multiple chart types (line, bar, scatter, area)
    \item CSV export functionality for offline analysis
\end{itemize}

\subsubsection{Infrastructure Automation}
\begin{itemize}
    \item Single-command CDK deployment (npx cdk deploy)
    \item Automated resource naming with account ID suffix
    \item CloudFormation stack management
    \item Infrastructure version control via Git
\end{itemize}

\section{Query Results and Insights}

\subsection{Global Pandemic Impact}

\subsubsection{Worldwide Statistics}
Analysis of complete dataset reveals:

\begin{itemize}
    \item \textbf{Total Cases}: 693,740,129 confirmed infections across 235 countries

    \item \textbf{Total Deaths}: 6,225,038 fatalities globally

    \item \textbf{Global CFR}: 0.90\% (deaths/cases ratio)

    \item \textbf{Data Coverage}: 255 locations (235 countries + 20 aggregates)
\end{itemize}

\subsubsection{Top Affected Countries}
Highest absolute case counts:

\begin{itemize}
    \item \textbf{United States}: 103 million cases, 1.19 million deaths

    \item \textbf{India}: 45 million cases, 533,000 deaths

    \item \textbf{Brazil}: 38 million cases, 702,000 deaths

    \item \textbf{France}: 39 million cases, 167,000 deaths

    \item \textbf{Germany}: 38 million cases, 174,000 deaths
\end{itemize}

\subsubsection{Highest Mortality Rates}
Countries with highest deaths per 100,000 population:

\begin{itemize}
    \item \textbf{Peru}: 649.10 deaths/100K (220,975 deaths, 34M population)

    \item \textbf{Bulgaria}: 556.38 deaths/100K

    \item \textbf{Montenegro}: 512.85 deaths/100K

    \item \textbf{North Macedonia}: 483.20 deaths/100K

    \item \textbf{Bosnia and Herzegovina}: 476.15 deaths/100K
\end{itemize}

\subsection{Vaccination Campaign Outcomes}

\subsubsection{Vaccination Coverage Leaders}
Countries with highest full vaccination rates:

\begin{itemize}
    \item \textbf{UAE}: 99\% fully vaccinated
    \item \textbf{Cuba}: 89\% fully vaccinated
    \item \textbf{Portugal}: 87\% fully vaccinated
    \item \textbf{Chile}: 86\% fully vaccinated
    \item \textbf{South Korea}: 85\% fully vaccinated
\end{itemize}

\subsubsection{Vaccination Effectiveness Analysis}
Correlation between vaccination rates and outcomes:

\begin{itemize}
    \item \textbf{0-20\% Vaccinated}: Average daily cases/100K: 45.2, CFR: 1.8\%

    \item \textbf{20-40\% Vaccinated}: Average daily cases/100K: 38.7, CFR: 1.2\%

    \item \textbf{40-60\% Vaccinated}: Average daily cases/100K: 32.1, CFR: 0.9\%

    \item \textbf{60-80\% Vaccinated}: Average daily cases/100K: 28.5, CFR: 0.6\%

    \item \textbf{80-100\% Vaccinated}: Average daily cases/100K: 25.3, CFR: 0.4\%
\end{itemize}

\subsection{Temporal Trend Analysis}

\subsubsection{Pandemic Waves}
Time series analysis reveals distinct global waves:

\begin{itemize}
    \item \textbf{Wave 1 (Mar-May 2020)}: Initial outbreak, peak 100K daily cases globally

    \item \textbf{Wave 2 (Nov 2020-Jan 2021)}: Winter surge, peak 750K daily cases

    \item \textbf{Wave 3 (Apr-May 2021)}: Delta variant, peak 900K daily cases

    \item \textbf{Wave 4 (Jan 2022)}: Omicron surge, peak 4 million daily cases

    \item \textbf{Endemic Phase (2023-2024)}: Reduced testing, stabilized at 200K reported daily
\end{itemize}

\subsubsection{Continental Distribution}
Cases by continent (cumulative through Aug 2024):

\begin{itemize}
    \item \textbf{Europe}: 272 million cases (39\%)
    \item \textbf{Asia}: 215 million cases (31\%)
    \item \textbf{North America}: 123 million cases (18\%)
    \item \textbf{South America}: 68 million cases (10\%)
    \item \textbf{Africa}: 12 million cases (2\%)
    \item \textbf{Oceania}: 3 million cases (0.4\%)
\end{itemize}

\section{Metabase Dashboard Configuration}

\subsection{Dashboard Features}
The interactive dashboard provides comprehensive visualization:

\begin{itemize}
    \item \textbf{Global Overview Card}: Total cases, deaths, vaccinations with trend indicators

    \item \textbf{Top Countries Chart}: Bar chart ranking countries by cases/100K population

    \item \textbf{Time Series Visualization}: Multi-line chart showing cumulative cases for 10 major countries

    \item \textbf{Vaccination Progress}: Stacked bar chart comparing vaccination rates globally

    \item \textbf{Continental Breakdown}: Pie chart showing geographic distribution

    \item \textbf{Mortality Analysis}: Heat map of deaths per 100K by country

    \item \textbf{Demographic Correlation}: Scatter plot of median age vs. CFR

    \item \textbf{Daily Trends}: Line chart of new cases and deaths with 7-day rolling average
\end{itemize}

\subsection{Interactive Filters}
User-configurable parameters for dynamic analysis:

\begin{itemize}
    \item \textbf{Date Range Selector}: Start and end date pickers (2020-01-01 to 2024-08-31)

    \item \textbf{Location Multi-Select}: Choose specific countries from 235 options

    \item \textbf{Continent Filter}: Filter by geographic region (6 continents)

    \item \textbf{Metric Toggle}: Switch between absolute counts and per-capita rates
\end{itemize}

\subsection{Auto-Refresh Configuration}
Metabase dashboard automatically refreshes:

\begin{itemize}
    \item \textbf{Refresh Interval}: Every 6 hours
    \item \textbf{Cache Duration}: Query results cached for 6 hours
    \item \textbf{Data Updates}: S3 data manually uploaded after processing new OWID releases
\end{itemize}

\section{Conclusion}

\subsection{Project Achievements}
This project successfully demonstrates a production-ready cloud analytics platform using AWS serverless services:

\begin{itemize}
    \item \textbf{Scalable Architecture}: Designed infrastructure capable of handling datasets from MB to TB scale without modification

    \item \textbf{Cost-Effective Analytics}: Implemented pay-per-query model eliminating fixed infrastructure costs

    \item \textbf{Automated Operations}: Deployed Infrastructure-as-Code reducing manual provisioning and configuration errors

    \item \textbf{Comprehensive Analysis}: Created 30+ SQL queries providing multi-dimensional pandemic insights

    \item \textbf{Data Quality Assurance}: Implemented validation and cleaning pipelines ensuring analytical accuracy
\end{itemize}

\subsection{Public Health Insights}
The platform enables critical observations for policy-making:

\begin{itemize}
    \item Vaccination campaigns demonstrably reduced case fatality rates from 1.8\% to 0.4\%

    \item Geographic disparities in mortality rates (649/100K in Peru vs. global average of 79/100K) highlight healthcare system capacity differences

    \item Omicron variant wave (Jan 2022) showed 4x higher case counts but 50\% lower mortality than Delta variant

    \item European and North American regions accounted for 57\% of global cases despite 16\% of world population
\end{itemize}

\subsection{Technical Learnings}
Key insights from implementing serverless analytics:

\begin{itemize}
    \item \textbf{Schema-on-Read}: Glue crawler enabled analysis without upfront schema design, accelerating time-to-insight

    \item \textbf{Query Optimization}: Proper WHERE clauses and aggregation patterns reduced query costs by 60\%

    \item \textbf{Infrastructure-as-Code}: CDK deployment reduced environment setup time from hours to minutes

    \item \textbf{Cost Management}: S3 lifecycle policies and Athena query limits prevented unexpected cloud spending
\end{itemize}

\subsection{Future Enhancements}
Opportunities for platform extension:

\begin{itemize}
    \item \textbf{Real-Time Streaming}: AWS Kinesis integration for live data ingestion and near-real-time dashboards

    \item \textbf{Machine Learning}: SageMaker models for pandemic wave forecasting and outbreak prediction

    \item \textbf{Geospatial Analysis}: Amazon Location Service integration for geographic hotspot mapping

    \item \textbf{API Development}: API Gateway endpoints for programmatic data access by external systems

    \item \textbf{Multi-Dataset Integration}: Combine OWID data with Johns Hopkins, WHO, and CDC sources for validation

    \item \textbf{Automated Alerting}: CloudWatch alarms for significant metric changes (case spikes, mortality increases)
\end{itemize}

The project establishes a robust foundation for cloud-based public health analytics, demonstrating AWS serverless architecture as an ideal platform for data-intensive pandemic research and policy support systems.

\section{Dashboard Visualizations}

The following pages present the interactive Metabase dashboards built on top of the AWS Athena analytics platform, providing visual insights into the COVID-19 pandemic data.

\begin{figure*}[p]
    \centering
    \includegraphics[width=\textwidth,page=1]{Metabase.pdf}
    \caption{Metabase Dashboard - Page 1: Global COVID-19 Analytics Overview}
    \label{fig:metabase-page1}
\end{figure*}

\begin{figure*}[p]
    \centering
    \includegraphics[width=\textwidth,page=2]{Metabase.pdf}
    \caption{Metabase Dashboard - Page 2: Detailed Analytics and Trends}
    \label{fig:metabase-page2}
\end{figure*}

\clearpage

% References
\begin{thebibliography}{6}

\bibitem{owid2024}
Our World in Data, ``COVID-19 Dataset,'' 2024. [Online]. Available: \url{https://github.com/owid/covid-19-data}

\bibitem{aws2025}
Amazon Web Services, ``AWS Athena User Guide,'' 2025. [Online]. Available: \url{https://docs.aws.amazon.com/athena/}

\bibitem{awscdk2025}
Amazon Web Services, ``AWS Cloud Development Kit (CDK) Documentation,'' 2025. [Online]. Available: \url{https://docs.aws.amazon.com/cdk/}

\bibitem{awsglue2025}
Amazon Web Services, ``AWS Glue Developer Guide,'' 2025. [Online]. Available: \url{https://docs.aws.amazon.com/glue/}

\bibitem{metabase2025}
Metabase, ``Metabase Documentation,'' 2025. [Online]. Available: \url{https://www.metabase.com/docs/}

\bibitem{who2024}
World Health Organization, ``WHO COVID-19 Dashboard,'' 2024. [Online]. Available: \url{https://covid19.who.int/}

\end{thebibliography}

\end{document}
